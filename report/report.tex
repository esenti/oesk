\documentclass{article}

\usepackage[polish]{babel}
\usepackage[utf8]{inputenc}
\usepackage[T1]{fontenc}
\usepackage{graphicx}
\usepackage{float}

\begin{document}
\begin{center}
\textsc{\huge Ocena efektywności systemów komputerowych}


\vspace{0.8cm}
{\Large Badanie efektywności mechanizmów programowania obiektowego w języku C++}
\vspace{0.8cm}

Krzysztof Skoracki
\vspace{1cm}
\end{center}
\section{Wstęp}
Celem niniejszego ćwiczenia było zbadanie narzutu związanego z wykorzystywaniem mechanizmów programowania obiektowego w języku C++. W ramach ćwiczenia dokonano także porównania wyżej wspomnianych mechanizmów z technikami programowania proceduralnego.

\subsection{Pomiar czasu}
Do pomiaru czasu wykonywania poszczególnych instrukcji wykorzystano funkcję \texttt{std::chrono::high\_resolution\_clock::now()} zawartą w bibliotece standardowej języka C++ (od wersji C++11). Funkcja ta zapewnia użycie najdokładniejszego zegara dostępnego na danej platformie.

\subsection{Środowisko testowe}
Wszystkie testy były uruchamiane na komputerze z dwurdzeniowym procesorem Intel taktowanym zegarem 2.4GHz z 4GB pamięci RAM. Uzytym systemem operacyjnym było Ubuntu 12.10 w wersji 64-bitowej.

Program testowy kompilowany był za pomocą kompilatora \texttt{g++} w wersji 4.7.2 z wszystkimi dostępnymi optymalizacjami wyłączonymi:

\texttt{g++ -std=c++11 -O0 main.cpp}

\section{Przebieg ćwiczenia}
W ramach
\subsection{Koszty wywołania metody}
\begin{tabular}{ c c }
  \textbf{Metoda} & \textbf{Czas [ms]} \\ \hline
  pusta pętla & 2 \\ \hline
  wolna funkcja & 5 \\ \hline
  metoda & 8 \\ \hline
  metoda wirtualna & 8 \\
\end{tabular}
\subsection{Instrukcja warunkowa a polimorfizm}

% \begin{figure}[H]
% 	\centering
% 	\includegraphics[width=10cm]{tri.png}
% 	\caption{Widok planszy, na której ma miejsce gra.}
% \end{figure}

\section{Wnioski}
\end{document}
