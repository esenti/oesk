\documentclass{article}

\usepackage[polish]{babel}
\usepackage[utf8]{inputenc}
\usepackage[T1]{fontenc}
\usepackage{graphicx}
\usepackage{float}

\begin{document}
\renewcommand*{\tablename}{Tab.}
\begin{center}
\textsc{\huge Ocena efektywności systemów komputerowych}


\vspace{0.8cm}
{\Large Badanie efektywności mechanizmów programowania obiektowego w języku C++}
\vspace{0.8cm}

Krzysztof Skoracki
\vspace{1cm}
\end{center}
\section{Wstęp}
Celem niniejszego ćwiczenia było zbadanie narzutu związanego z wykorzystaniem mechanizmów programowania obiektowego w języku C++. W ramach badania dokonano także porównania wyżej wspomnianych mechanizmów z technikami programowania proceduralnego.

\subsection{Pomiar czasu}
Do pomiaru czasu wykonywania poszczególnych instrukcji wykorzystano funkcję \texttt{std::chrono::high\_resolution\_clock::now()} zawartą w bibliotece standardowej języka C++ (od wersji C++11). Funkcja ta zapewnia użycie najdokładniejszego zegara dostępnego na danej platformie.

\subsection{Środowisko testowe}
Wszystkie testy były uruchamiane na komputerze z dwurdzeniowym procesorem Intel taktowanym zegarem 2.4GHz z 4GB pamięci RAM. Użytym systemem operacyjnym było Ubuntu 12.10 w wersji 64-bitowej.

Program testowy kompilowany był za pomocą kompilatora \texttt{g++} w wersji 4.7.2 z wszystkimi dostępnymi optymalizacjami wyłączonymi:

\texttt{g++ -std=c++11 -O0 main.cpp}

\section{Przebieg badania}
Testowane funkcje wywoływane były w pętli po 2000000000 iteracji. Dodatkowo program testowy uruchamiany był 10 razy -- wartości w tabelach stanowią wartość średnią ze wszystkich uruchomień.

\subsection{Koszty wywołania funkcji i metody}
Celem tego badania było porównanie kosztu wywołania kilku rodzajów funkcji:
\begin{itemize}
\item funkcja -- wolna funkcja
\item metoda -- metoda klasy
\item metoda wirtualna -- metoda oznaczona jako wirtualna wywoływana dla referencji do klasy bazowej, do której przypisany został obiekt klasy pochodnej: \texttt{const VirtualBase\& testVirtual = VirtualChild();}
\end{itemize}

Wyniki przedstawiono w tabeli \ref{table:fun}.

\begin{table}[H]
\center
\begin{tabular}{ c c c }
  \textbf{Metoda} & \textbf{Czas [ms]} & \textbf{Czas iteracji [ns]} \\ \hline
  funkcja & 7005 & 3,5 \\ \hline
  metoda & 8074 & 4,04\\ \hline
  metoda wirtualna & 9229 & 4,61 \\
\end{tabular}
\caption{Porównanie kosztu wywołania różnych rodzajów funkcji}
\label{table:fun}
\end{table}

Zgodnie z oczekiwaniami, wywołanie metody klasy jest nieco bardziej kosztowne niż wywołanie wolnej funkcji. Jest to zapewne związane z koniecznością przekazania dodatkowego parametru wywołania -- wskaźnika do obiektu, dla którego metoda jest wywoływana.

Kolejny narzut związany jest z wywołaniem metody wirtualnej -- konieczne jest w tym przypadku wybranie właściwej funkcji korzystając z tablicy funkcji wirtualnych dla danej klasy (\textit{vtable}).

Warto zauważyć, że mimo, iż różnica wydajności jest widoczna w testach, to narzut w wysokości 1ns nie powinien być problemem w większości praktycznych zastosowań.

\subsection{Instrukcja warunkowa a polimorfizm}

W celu zbadania wydajności mechanizmu polimorfizmu w nieco bardziej praktycznych warunkach, przeprowadzono kolejne badanie. Miało ono na celu określenie, jakie są różnice wydajności pomiędzy wywołaniem funkcji wirtualnej a zastosowaniem zwykłej instrukcji warunkowej.

Funkcja \texttt{freeFunction} zwraca wartość 5, natomiast funkcja \texttt{freeFunction2} -- 1.

\begin{itemize}
\item warunek, wariant 1 -- \texttt{if(f == 5) f = 1; else f = 5;}
\item warunek, wariant 2 -- \texttt{if(f == 1) f = 1; else f = 5;}
\item warunek + funkcja, wariant 1 -- \texttt{if(f == 5) f = freeFunction2(); else f = freeFunction();}
\item warunek + funkcja, wariant 2 -- \texttt{if(f == 1) f = freeFunction2(); else f = freeFunction();}
\item metoda wirtualna -- \texttt{f = testVirtual.virtualMethod();}
\end{itemize}

Wyniki przedstawiono w tabeli \ref{table:if}.

\begin{table}[H]
\center
\begin{tabular}{ c c c }
  \textbf{Metoda} & \textbf{Czas [ms]} & \textbf{Czas iteracji [ns]} \\ \hline
  warunek, wariant 1 & 7336 & 3,67 \\ \hline
  warunek, wariant 2 & 7277 & 3,64 \\ \hline
  warunek + funkcja, wariant 1 & 9592 & 4,8 \\ \hline
  warunek + funkcja, wariant 2 & 9111 & 4,56 \\ \hline
  metoda wirtualna & 9366 & 4,68 \\
\end{tabular}
\caption{Porównanie kosztu wywołania funkcji wirtualnej oraz zastosowania instrukcji warunkowej}
\label{table:if}
\end{table}

Zgodnie z przewidywaniami, wersja z bezpośrednim wywołaniem instrukcji okazała się najszybsza. Z kolei w przypadku instrukcji warunkowej wywołującej odpowiednią funkcję, jej koszty są zbliżone do kosztów wywołania metody wirtualnej. Różnica na korzyść jednej z nich zależy od wybranego wariantu tej pierwszej.

Warto zwrócić uwagę na różnicę wydajności pomiędzy dwoma wariantami w przypadku instrukcji warunkowej. Różnica ta związana jest z mechanizmem przewidywania rozgałęzień (\textit{branch prediction}) w procesorze. W przypadku wariantu 2 warunek w każdej iteracji ma taką samą wartość, co sprawia, że przewidzenie, która gałąź zostanie wybrana jest trywialne. Dla wariantu 1 warunek przybiera dwie wartości, zmieniające się cyklicznie w każdej iteracji -- jak widać, utrudnia to działanie mechanizmu przewidywania rozgałęzień.

Podobnie jak w pierwszym badaniu, i tutaj różnice wydajności poszczególnych rozwiązań są raczej niewielkie. Na szczególną uwagę zasługuje fakt, że rozwiązanie używające instrukcji warunkowej oraz wywołania metody wirtualnej mają niemal identyczny koszt.

\section{Wnioski}
Jak widać, narzut związany z wykorzystaniem pewnych technik programowania obiektowego w języku C++ istnieje i udało się go zaobserwować w niniejszym badaniu. Co ważniejsze jednak, wyniki pokazują, że różnice w wydajności, mimo, że zauważalne w warunkach testowych, w zastosowaniach praktycznych najczęściej nie będą miały znaczenia.

\end{document}
